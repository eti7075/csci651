%% Generated by Sphinx.
\def\sphinxdocclass{report}
\documentclass[letterpaper,10pt,english,openany,oneside]{sphinxmanual}
\ifdefined\pdfpxdimen
   \let\sphinxpxdimen\pdfpxdimen\else\newdimen\sphinxpxdimen
\fi \sphinxpxdimen=.75bp\relax
\ifdefined\pdfimageresolution
    \pdfimageresolution= \numexpr \dimexpr1in\relax/\sphinxpxdimen\relax
\fi
%% let collapsible pdf bookmarks panel have high depth per default
\PassOptionsToPackage{bookmarksdepth=5}{hyperref}

\PassOptionsToPackage{booktabs}{sphinx}
\PassOptionsToPackage{colorrows}{sphinx}

\PassOptionsToPackage{warn}{textcomp}
\usepackage[utf8]{inputenc}
\ifdefined\DeclareUnicodeCharacter
% support both utf8 and utf8x syntaxes
  \ifdefined\DeclareUnicodeCharacterAsOptional
    \def\sphinxDUC#1{\DeclareUnicodeCharacter{"#1}}
  \else
    \let\sphinxDUC\DeclareUnicodeCharacter
  \fi
  \sphinxDUC{00A0}{\nobreakspace}
  \sphinxDUC{2500}{\sphinxunichar{2500}}
  \sphinxDUC{2502}{\sphinxunichar{2502}}
  \sphinxDUC{2514}{\sphinxunichar{2514}}
  \sphinxDUC{251C}{\sphinxunichar{251C}}
  \sphinxDUC{2572}{\textbackslash}
\fi
\usepackage{cmap}
\usepackage[T1]{fontenc}
\usepackage{amsmath,amssymb,amstext}
\usepackage{babel}



\usepackage{tgtermes}
\usepackage{tgheros}
\renewcommand{\ttdefault}{txtt}



\usepackage[Bjarne]{fncychap}
\usepackage{sphinx}

\fvset{fontsize=auto}
\usepackage{geometry}


% Include hyperref last.
\usepackage{hyperref}
% Fix anchor placement for figures with captions.
\usepackage{hypcap}% it must be loaded after hyperref.
% Set up styles of URL: it should be placed after hyperref.
\urlstyle{same}

\addto\captionsenglish{\renewcommand{\contentsname}{Contents:}}

\usepackage{sphinxmessages}
\setcounter{tocdepth}{1}



\title{Mininet}
\date{Apr 19, 2025}
\release{}
\author{Ethan Iannicelli}
\newcommand{\sphinxlogo}{\vbox{}}
\renewcommand{\releasename}{}
\makeindex
\begin{document}

\ifdefined\shorthandoff
  \ifnum\catcode`\=\string=\active\shorthandoff{=}\fi
  \ifnum\catcode`\"=\active\shorthandoff{"}\fi
\fi

\pagestyle{empty}
\sphinxmaketitle
\pagestyle{plain}
\sphinxtableofcontents
\pagestyle{normal}
\phantomsection\label{\detokenize{index::doc}}



\chapter{Mininet network}
\label{\detokenize{index:mininet-network}}
\sphinxAtStartPar
The \sphinxcode{\sphinxupquote{LinuxRouter}} class represents a router that is a part of the network.
Notably, the net.ipv4.ip\_forward value is set to 1 on initialization.
\index{LinuxRouter (class in layer3\_network\_code)@\spxentry{LinuxRouter}\spxextra{class in layer3\_network\_code}}

\begin{fulllineitems}
\phantomsection\label{\detokenize{index:layer3_network_code.LinuxRouter}}
\pysigstartsignatures
\pysigline
{\sphinxbfcode{\sphinxupquote{class\DUrole{w}{ }}}\sphinxcode{\sphinxupquote{layer3\_network\_code.}}\sphinxbfcode{\sphinxupquote{LinuxRouter}}}
\pysigstopsignatures
\sphinxAtStartPar
Bases: \sphinxcode{\sphinxupquote{object}}
\index{config() (layer3\_network\_code.LinuxRouter method)@\spxentry{config()}\spxextra{layer3\_network\_code.LinuxRouter method}}

\begin{fulllineitems}
\phantomsection\label{\detokenize{index:layer3_network_code.LinuxRouter.config}}
\pysigstartsignatures
\pysiglinewithargsret
{\sphinxbfcode{\sphinxupquote{config}}}
{\sphinxparam{\DUrole{o}{**}\DUrole{n}{params}}}
{}
\pysigstopsignatures
\sphinxAtStartPar
Sets the configuration for the router. sets ipv4 forwarding to true.

\end{fulllineitems}

\index{terminate() (layer3\_network\_code.LinuxRouter method)@\spxentry{terminate()}\spxextra{layer3\_network\_code.LinuxRouter method}}

\begin{fulllineitems}
\phantomsection\label{\detokenize{index:layer3_network_code.LinuxRouter.terminate}}
\pysigstartsignatures
\pysiglinewithargsret
{\sphinxbfcode{\sphinxupquote{terminate}}}
{}
{}
\pysigstopsignatures
\sphinxAtStartPar
Run on router ternimation. sets ipv4 forwarding to false

\end{fulllineitems}


\end{fulllineitems}


\sphinxAtStartPar
Our network topography is built in the \sphinxcode{\sphinxupquote{NetworkTopo}} class. This class extends
the mininet \sphinxcode{\sphinxupquote{Topo}} class, and overrides the \sphinxcode{\sphinxupquote{build}} function
for this program.
\index{NetworkTopo (class in layer3\_network\_code)@\spxentry{NetworkTopo}\spxextra{class in layer3\_network\_code}}

\begin{fulllineitems}
\phantomsection\label{\detokenize{index:layer3_network_code.NetworkTopo}}
\pysigstartsignatures
\pysigline
{\sphinxbfcode{\sphinxupquote{class\DUrole{w}{ }}}\sphinxcode{\sphinxupquote{layer3\_network\_code.}}\sphinxbfcode{\sphinxupquote{NetworkTopo}}}
\pysigstopsignatures
\sphinxAtStartPar
Bases: \sphinxcode{\sphinxupquote{object}}
\index{build() (layer3\_network\_code.NetworkTopo method)@\spxentry{build()}\spxextra{layer3\_network\_code.NetworkTopo method}}

\begin{fulllineitems}
\phantomsection\label{\detokenize{index:layer3_network_code.NetworkTopo.build}}
\pysigstartsignatures
\pysiglinewithargsret
{\sphinxbfcode{\sphinxupquote{build}}}
{\sphinxparam{\DUrole{o}{**}\DUrole{n}{\_opts}}}
{}
\pysigstopsignatures
\sphinxAtStartPar
Builds the network outlined for this homework. Three LANs represented by routers, each connected to
2 hosts via switches, and all routers connected through the provided IP gateway

\end{fulllineitems}


\end{fulllineitems}


\sphinxAtStartPar
To build, run and create routes between nodes, use the \sphinxcode{\sphinxupquote{run}} function. This
function also starts the mininet CLI and cleans up the network on exiting.
\index{run() (in module layer3\_network\_code)@\spxentry{run()}\spxextra{in module layer3\_network\_code}}

\begin{fulllineitems}
\phantomsection\label{\detokenize{index:layer3_network_code.run}}
\pysigstartsignatures
\pysiglinewithargsret
{\sphinxcode{\sphinxupquote{layer3\_network\_code.}}\sphinxbfcode{\sphinxupquote{run}}}
{}
{}
\pysigstopsignatures
\sphinxAtStartPar
Create network for this assignment. Start the network running, then open the mininet CLI to support
running commands and examing network properties. Close the network when the CLI is exited.

\end{fulllineitems}




\renewcommand{\indexname}{Index}
\printindex
\end{document}