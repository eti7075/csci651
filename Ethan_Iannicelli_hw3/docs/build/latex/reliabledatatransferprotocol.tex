%% Generated by Sphinx.
\def\sphinxdocclass{report}
\documentclass[letterpaper,10pt,english,openany,oneside]{sphinxmanual}
\ifdefined\pdfpxdimen
   \let\sphinxpxdimen\pdfpxdimen\else\newdimen\sphinxpxdimen
\fi \sphinxpxdimen=.75bp\relax
\ifdefined\pdfimageresolution
    \pdfimageresolution= \numexpr \dimexpr1in\relax/\sphinxpxdimen\relax
\fi
%% let collapsible pdf bookmarks panel have high depth per default
\PassOptionsToPackage{bookmarksdepth=5}{hyperref}

\PassOptionsToPackage{booktabs}{sphinx}
\PassOptionsToPackage{colorrows}{sphinx}

\PassOptionsToPackage{warn}{textcomp}
\usepackage[utf8]{inputenc}
\ifdefined\DeclareUnicodeCharacter
% support both utf8 and utf8x syntaxes
  \ifdefined\DeclareUnicodeCharacterAsOptional
    \def\sphinxDUC#1{\DeclareUnicodeCharacter{"#1}}
  \else
    \let\sphinxDUC\DeclareUnicodeCharacter
  \fi
  \sphinxDUC{00A0}{\nobreakspace}
  \sphinxDUC{2500}{\sphinxunichar{2500}}
  \sphinxDUC{2502}{\sphinxunichar{2502}}
  \sphinxDUC{2514}{\sphinxunichar{2514}}
  \sphinxDUC{251C}{\sphinxunichar{251C}}
  \sphinxDUC{2572}{\textbackslash}
\fi
\usepackage{cmap}
\usepackage[T1]{fontenc}
\usepackage{amsmath,amssymb,amstext}
\usepackage{babel}



\usepackage{tgtermes}
\usepackage{tgheros}
\renewcommand{\ttdefault}{txtt}



\usepackage[Bjarne]{fncychap}
\usepackage{sphinx}

\fvset{fontsize=auto}
\usepackage{geometry}


% Include hyperref last.
\usepackage{hyperref}
% Fix anchor placement for figures with captions.
\usepackage{hypcap}% it must be loaded after hyperref.
% Set up styles of URL: it should be placed after hyperref.
\urlstyle{same}

\addto\captionsenglish{\renewcommand{\contentsname}{Contents:}}

\usepackage{sphinxmessages}
\setcounter{tocdepth}{1}



\title{Reliable Data Transfer Protocol}
\date{Mar 14, 2025}
\release{1.0}
\author{Ethan Iannicelli}
\newcommand{\sphinxlogo}{\vbox{}}
\renewcommand{\releasename}{Release}
\makeindex
\begin{document}

\ifdefined\shorthandoff
  \ifnum\catcode`\=\string=\active\shorthandoff{=}\fi
  \ifnum\catcode`\"=\active\shorthandoff{"}\fi
\fi

\pagestyle{empty}
\sphinxmaketitle
\pagestyle{plain}
\sphinxtableofcontents
\pagestyle{normal}
\phantomsection\label{\detokenize{index::doc}}



\chapter{RDT Protocol}
\label{\detokenize{index:rdt-protocol}}
\sphinxAtStartPar
To get the checksum of an ICMP packet based on the string representation of the packet, use the \sphinxcode{\sphinxupquote{udp\_checksum()}} function:
\index{udp\_checksum() (in module rdt\_protocol)@\spxentry{udp\_checksum()}\spxextra{in module rdt\_protocol}}

\begin{fulllineitems}
\phantomsection\label{\detokenize{index:rdt_protocol.udp_checksum}}
\pysigstartsignatures
\pysiglinewithargsret
{\sphinxcode{\sphinxupquote{rdt\_protocol.}}\sphinxbfcode{\sphinxupquote{udp\_checksum}}}
{\sphinxparam{\DUrole{n}{data}}}
{}
\pysigstopsignatures
\sphinxAtStartPar
perform a psuedo udp checksum by reducing the data to 4 bytes and taking one’s complement
\begin{quote}\begin{description}
\sphinxlineitem{Parameters}
\sphinxAtStartPar
\sphinxstyleliteralstrong{\sphinxupquote{data}} (\sphinxstyleliteralemphasis{\sphinxupquote{bitstring}}) \textendash{} the data that the checksum is created from

\sphinxlineitem{Returns}
\sphinxAtStartPar
generated checksum

\sphinxlineitem{Return type}
\sphinxAtStartPar
int

\end{description}\end{quote}

\end{fulllineitems}


\sphinxAtStartPar
To create a packet using a sequence number, acknowledgment number, and data, use the \sphinxcode{\sphinxupquote{create\_packet()}} function:
\index{create\_packet() (in module rdt\_protocol)@\spxentry{create\_packet()}\spxextra{in module rdt\_protocol}}

\begin{fulllineitems}
\phantomsection\label{\detokenize{index:rdt_protocol.create_packet}}
\pysigstartsignatures
\pysiglinewithargsret
{\sphinxcode{\sphinxupquote{rdt\_protocol.}}\sphinxbfcode{\sphinxupquote{create\_packet}}}
{\sphinxparam{\DUrole{n}{seq\_num}}\sphinxparamcomma \sphinxparam{\DUrole{n}{ack\_num}}\sphinxparamcomma \sphinxparam{\DUrole{n}{data}}}
{}
\pysigstopsignatures
\sphinxAtStartPar
create a packet using a sequence number, ack number, and data
\begin{quote}\begin{description}
\sphinxlineitem{Parameters}\begin{itemize}
\item {} 
\sphinxAtStartPar
\sphinxstyleliteralstrong{\sphinxupquote{seq\_num}} (\sphinxstyleliteralemphasis{\sphinxupquote{int}}) \textendash{} the sequence number of the packet

\item {} 
\sphinxAtStartPar
\sphinxstyleliteralstrong{\sphinxupquote{ack\_num}} (\sphinxstyleliteralemphasis{\sphinxupquote{int}}) \textendash{} the ack number of the packet

\item {} 
\sphinxAtStartPar
\sphinxstyleliteralstrong{\sphinxupquote{data}} (\sphinxstyleliteralemphasis{\sphinxupquote{bitstring}}) \textendash{} the data to be included in the packet

\end{itemize}

\sphinxlineitem{Returns}
\sphinxAtStartPar
bitstring representing formed packet

\sphinxlineitem{Return type}
\sphinxAtStartPar
bitstring

\end{description}\end{quote}

\end{fulllineitems}


\sphinxAtStartPar
To parse a packet into its sequence number, acknowledgment number, checksum, and data, use the \sphinxcode{\sphinxupquote{parse\_packet()}} function:
\index{parse\_packet() (in module rdt\_protocol)@\spxentry{parse\_packet()}\spxextra{in module rdt\_protocol}}

\begin{fulllineitems}
\phantomsection\label{\detokenize{index:rdt_protocol.parse_packet}}
\pysigstartsignatures
\pysiglinewithargsret
{\sphinxcode{\sphinxupquote{rdt\_protocol.}}\sphinxbfcode{\sphinxupquote{parse\_packet}}}
{\sphinxparam{\DUrole{n}{packet}}}
{}
\pysigstopsignatures
\sphinxAtStartPar
extracts seq, ack, checksum, and data from a packet
\begin{quote}\begin{description}
\sphinxlineitem{Parameters}
\sphinxAtStartPar
\sphinxstyleliteralstrong{\sphinxupquote{packet}} (\sphinxstyleliteralemphasis{\sphinxupquote{bitstring}}) \textendash{} formatted packet

\sphinxlineitem{Returns}
\sphinxAtStartPar
4 tuple of seq, ack, check, data

\sphinxlineitem{Return type}
\sphinxAtStartPar
tuple

\end{description}\end{quote}

\end{fulllineitems}


\sphinxAtStartPar
To split a bitstring of data into multiple parts of a given size, use the \sphinxcode{\sphinxupquote{split\_data()}} function:
\index{split\_data() (in module rdt\_protocol)@\spxentry{split\_data()}\spxextra{in module rdt\_protocol}}

\begin{fulllineitems}
\phantomsection\label{\detokenize{index:rdt_protocol.split_data}}
\pysigstartsignatures
\pysiglinewithargsret
{\sphinxcode{\sphinxupquote{rdt\_protocol.}}\sphinxbfcode{\sphinxupquote{split\_data}}}
{\sphinxparam{\DUrole{n}{data}}\sphinxparamcomma \sphinxparam{\DUrole{n}{chunk\_size}}}
{}
\pysigstopsignatures
\sphinxAtStartPar
splits a bitstring of data into multiple parts of a given size
\begin{quote}\begin{description}
\sphinxlineitem{Parameters}\begin{itemize}
\item {} 
\sphinxAtStartPar
\sphinxstyleliteralstrong{\sphinxupquote{data}} (\sphinxstyleliteralemphasis{\sphinxupquote{bitstring}}) \textendash{} bitstring of the full data

\item {} 
\sphinxAtStartPar
\sphinxstyleliteralstrong{\sphinxupquote{chunk\_size}} (\sphinxstyleliteralemphasis{\sphinxupquote{int}}) \textendash{} maximum chunk size

\end{itemize}

\sphinxlineitem{Returns}
\sphinxAtStartPar
array of data split up into chunk\_sizes

\sphinxlineitem{Return type}
\sphinxAtStartPar
array

\end{description}\end{quote}

\end{fulllineitems}


\sphinxAtStartPar
The \sphinxcode{\sphinxupquote{ReliableDataTransferEntity}} class represents an RDT entity that can act as a sender or receiver:
\index{ReliableDataTransferEntity (class in rdt\_protocol)@\spxentry{ReliableDataTransferEntity}\spxextra{class in rdt\_protocol}}

\begin{fulllineitems}
\phantomsection\label{\detokenize{index:rdt_protocol.ReliableDataTransferEntity}}
\pysigstartsignatures
\pysiglinewithargsret
{\sphinxbfcode{\sphinxupquote{class\DUrole{w}{ }}}\sphinxcode{\sphinxupquote{rdt\_protocol.}}\sphinxbfcode{\sphinxupquote{ReliableDataTransferEntity}}}
{\sphinxparam{\DUrole{n}{inter\_address}}\sphinxparamcomma \sphinxparam{\DUrole{n}{entity\_address}}\sphinxparamcomma \sphinxparam{\DUrole{n}{window\_size}\DUrole{o}{=}\DUrole{default_value}{4}}\sphinxparamcomma \sphinxparam{\DUrole{n}{timeout}\DUrole{o}{=}\DUrole{default_value}{True}}}
{}
\pysigstopsignatures
\sphinxAtStartPar
Bases: \sphinxcode{\sphinxupquote{object}}

\sphinxAtStartPar
class for a RDT entity, either a client or server. different types of entity are differentiated by
their actions and behaviors
\index{receive() (rdt\_protocol.ReliableDataTransferEntity method)@\spxentry{receive()}\spxextra{rdt\_protocol.ReliableDataTransferEntity method}}

\begin{fulllineitems}
\phantomsection\label{\detokenize{index:rdt_protocol.ReliableDataTransferEntity.receive}}
\pysigstartsignatures
\pysiglinewithargsret
{\sphinxbfcode{\sphinxupquote{receive}}}
{}
{}
\pysigstopsignatures
\sphinxAtStartPar
recieves data from a network
\begin{quote}\begin{description}
\sphinxlineitem{Parameters}
\sphinxAtStartPar
\sphinxstyleliteralstrong{\sphinxupquote{self}} ({\hyperref[\detokenize{index:rdt_protocol.ReliableDataTransferEntity}]{\sphinxcrossref{\sphinxstyleliteralemphasis{\sphinxupquote{ReliableDataTransferEntity}}}}}) \textendash{} the receiver object

\sphinxlineitem{Returns}
\sphinxAtStartPar
the data in the packet

\sphinxlineitem{Return type}
\sphinxAtStartPar
bitstring

\end{description}\end{quote}

\end{fulllineitems}

\index{send() (rdt\_protocol.ReliableDataTransferEntity method)@\spxentry{send()}\spxextra{rdt\_protocol.ReliableDataTransferEntity method}}

\begin{fulllineitems}
\phantomsection\label{\detokenize{index:rdt_protocol.ReliableDataTransferEntity.send}}
\pysigstartsignatures
\pysiglinewithargsret
{\sphinxbfcode{\sphinxupquote{send}}}
{\sphinxparam{\DUrole{n}{data}}}
{}
\pysigstopsignatures
\sphinxAtStartPar
sends data based on the entity of the sender
\begin{quote}\begin{description}
\sphinxlineitem{Parameters}\begin{itemize}
\item {} 
\sphinxAtStartPar
\sphinxstyleliteralstrong{\sphinxupquote{self}} ({\hyperref[\detokenize{index:rdt_protocol.ReliableDataTransferEntity}]{\sphinxcrossref{\sphinxstyleliteralemphasis{\sphinxupquote{ReliableDataTransferEntity}}}}}) \textendash{} the sender object

\item {} 
\sphinxAtStartPar
\sphinxstyleliteralstrong{\sphinxupquote{data}} (\sphinxstyleliteralemphasis{\sphinxupquote{bitstring}}) \textendash{} data to be sent

\end{itemize}

\end{description}\end{quote}

\end{fulllineitems}


\end{fulllineitems}


\sphinxAtStartPar
To simulate packet loss, use the \sphinxcode{\sphinxupquote{simulate\_loss()}} function:
\index{simulate\_loss() (in module intermediary)@\spxentry{simulate\_loss()}\spxextra{in module intermediary}}

\begin{fulllineitems}
\phantomsection\label{\detokenize{index:intermediary.simulate_loss}}
\pysigstartsignatures
\pysiglinewithargsret
{\sphinxcode{\sphinxupquote{intermediary.}}\sphinxbfcode{\sphinxupquote{simulate\_loss}}}
{\sphinxparam{\DUrole{n}{packet}}}
{}
\pysigstopsignatures
\sphinxAtStartPar
simulate packet loss
\begin{quote}\begin{description}
\sphinxlineitem{Returns}
\sphinxAtStartPar
the packet if no loss, None if else

\sphinxlineitem{Return type}
\sphinxAtStartPar
bitstring?

\end{description}\end{quote}

\end{fulllineitems}


\sphinxAtStartPar
To simulate packet corruption, use the \sphinxcode{\sphinxupquote{simulate\_corruption()}} function:
\index{simulate\_corruption() (in module intermediary)@\spxentry{simulate\_corruption()}\spxextra{in module intermediary}}

\begin{fulllineitems}
\phantomsection\label{\detokenize{index:intermediary.simulate_corruption}}
\pysigstartsignatures
\pysiglinewithargsret
{\sphinxcode{\sphinxupquote{intermediary.}}\sphinxbfcode{\sphinxupquote{simulate\_corruption}}}
{\sphinxparam{\DUrole{n}{packet}}}
{}
\pysigstopsignatures
\sphinxAtStartPar
simulate packet curruption
\begin{quote}\begin{description}
\sphinxlineitem{Returns}
\sphinxAtStartPar
the packet

\sphinxlineitem{Return type}
\sphinxAtStartPar
bitstring

\end{description}\end{quote}

\end{fulllineitems}


\sphinxAtStartPar
To simulate packet reordering in the packet queue, use the \sphinxcode{\sphinxupquote{simulate\_reordering()}} function:
\index{simulate\_reordering() (in module intermediary)@\spxentry{simulate\_reordering()}\spxextra{in module intermediary}}

\begin{fulllineitems}
\phantomsection\label{\detokenize{index:intermediary.simulate_reordering}}
\pysigstartsignatures
\pysiglinewithargsret
{\sphinxcode{\sphinxupquote{intermediary.}}\sphinxbfcode{\sphinxupquote{simulate\_reordering}}}
{\sphinxparam{\DUrole{n}{packet\_queue}}}
{}
\pysigstopsignatures
\sphinxAtStartPar
simulate packet queue reordering
\begin{quote}\begin{description}
\sphinxlineitem{Returns}
\sphinxAtStartPar
packet queue

\sphinxlineitem{Return type}
\sphinxAtStartPar
array

\end{description}\end{quote}

\end{fulllineitems}


\sphinxAtStartPar
To simulate packet delay via sleep, use the \sphinxcode{\sphinxupquote{simulate\_delay()}} function:
\index{simulate\_delay() (in module intermediary)@\spxentry{simulate\_delay()}\spxextra{in module intermediary}}

\begin{fulllineitems}
\phantomsection\label{\detokenize{index:intermediary.simulate_delay}}
\pysigstartsignatures
\pysiglinewithargsret
{\sphinxcode{\sphinxupquote{intermediary.}}\sphinxbfcode{\sphinxupquote{simulate\_delay}}}
{}
{}
\pysigstopsignatures
\sphinxAtStartPar
simulate packet delay via sleep

\end{fulllineitems}


\sphinxAtStartPar
To handle a packet by applying network conditions and forwarding it to an address, use the \sphinxcode{\sphinxupquote{handle\_packet()}} function:
\index{handle\_packet() (in module intermediary)@\spxentry{handle\_packet()}\spxextra{in module intermediary}}

\begin{fulllineitems}
\phantomsection\label{\detokenize{index:intermediary.handle_packet}}
\pysigstartsignatures
\pysiglinewithargsret
{\sphinxcode{\sphinxupquote{intermediary.}}\sphinxbfcode{\sphinxupquote{handle\_packet}}}
{\sphinxparam{\DUrole{n}{packet}}\sphinxparamcomma \sphinxparam{\DUrole{n}{packet\_queue}}\sphinxparamcomma \sphinxparam{\DUrole{n}{inter\_socket}}\sphinxparamcomma \sphinxparam{\DUrole{n}{forward\_address}}}
{}
\pysigstopsignatures
\sphinxAtStartPar
handles a packet by undergoing network conditions and forwarding to address
\begin{quote}\begin{description}
\sphinxlineitem{Parameters}\begin{itemize}
\item {} 
\sphinxAtStartPar
\sphinxstyleliteralstrong{\sphinxupquote{packet}} (\sphinxstyleliteralemphasis{\sphinxupquote{bitstring}}) \textendash{} the packet to be handled

\item {} 
\sphinxAtStartPar
\sphinxstyleliteralstrong{\sphinxupquote{packet\_queue}} (\sphinxstyleliteralemphasis{\sphinxupquote{array}}) \textendash{} queue of packets to be delivered

\item {} 
\sphinxAtStartPar
\sphinxstyleliteralstrong{\sphinxupquote{inter\_socket}} \textendash{} the socket of this script

\item {} 
\sphinxAtStartPar
\sphinxstyleliteralstrong{\sphinxupquote{forward\_address}} (\sphinxstyleliteralemphasis{\sphinxupquote{2 tuple}}\sphinxstyleliteralemphasis{\sphinxupquote{ of }}\sphinxstyleliteralemphasis{\sphinxupquote{ip and port}}) \textendash{} the address to forward the packet to

\end{itemize}

\end{description}\end{quote}

\end{fulllineitems}


\sphinxAtStartPar
To run the intermediary that simulates network conditions and handles forwarding of packets, use the \sphinxcode{\sphinxupquote{run\_intermediary()}} function:
\index{run\_intermediary() (in module intermediary)@\spxentry{run\_intermediary()}\spxextra{in module intermediary}}

\begin{fulllineitems}
\phantomsection\label{\detokenize{index:intermediary.run_intermediary}}
\pysigstartsignatures
\pysiglinewithargsret
{\sphinxcode{\sphinxupquote{intermediary.}}\sphinxbfcode{\sphinxupquote{run\_intermediary}}}
{}
{}
\pysigstopsignatures
\sphinxAtStartPar
runs the intermediary that acts as a network for this project. simulates network conditions
and handles forwarding of packets

\end{fulllineitems}


\sphinxAtStartPar
To send all data from a given file to the server, use the \sphinxcode{\sphinxupquote{send\_file()}} function:

\sphinxAtStartPar
The \sphinxcode{\sphinxupquote{FileTransferClient}} class represents a client for the file transfer procedure:
\index{FileTransferClient (class in client)@\spxentry{FileTransferClient}\spxextra{class in client}}

\begin{fulllineitems}
\phantomsection\label{\detokenize{index:client.FileTransferClient}}
\pysigstartsignatures
\pysigline
{\sphinxbfcode{\sphinxupquote{class\DUrole{w}{ }}}\sphinxcode{\sphinxupquote{client.}}\sphinxbfcode{\sphinxupquote{FileTransferClient}}}
\pysigstopsignatures
\sphinxAtStartPar
Bases: \sphinxcode{\sphinxupquote{object}}
\index{send\_file() (client.FileTransferClient method)@\spxentry{send\_file()}\spxextra{client.FileTransferClient method}}

\begin{fulllineitems}
\phantomsection\label{\detokenize{index:client.FileTransferClient.send_file}}
\pysigstartsignatures
\pysiglinewithargsret
{\sphinxbfcode{\sphinxupquote{send\_file}}}
{\sphinxparam{\DUrole{n}{file\_path}}}
{}
\pysigstopsignatures
\sphinxAtStartPar
sends all the data from a given filepath to the server
\begin{quote}\begin{description}
\sphinxlineitem{Parameters}\begin{itemize}
\item {} 
\sphinxAtStartPar
\sphinxstyleliteralstrong{\sphinxupquote{self}} ({\hyperref[\detokenize{index:client.FileTransferClient}]{\sphinxcrossref{\sphinxstyleliteralemphasis{\sphinxupquote{FileTransferClient}}}}}) \textendash{} client in the file transfer procedure

\item {} 
\sphinxAtStartPar
\sphinxstyleliteralstrong{\sphinxupquote{file\_path}} (\sphinxstyleliteralemphasis{\sphinxupquote{String}}) \textendash{} relative filename to this program being run

\end{itemize}

\end{description}\end{quote}

\end{fulllineitems}


\end{fulllineitems}


\sphinxAtStartPar
To receive a file and save it to a designated folder, use the \sphinxcode{\sphinxupquote{receive\_file()}} function:

\sphinxAtStartPar
The \sphinxcode{\sphinxupquote{FileTransferServer}} class represents a server for receiving and saving files:
\index{FileTransferServer (class in server)@\spxentry{FileTransferServer}\spxextra{class in server}}

\begin{fulllineitems}
\phantomsection\label{\detokenize{index:server.FileTransferServer}}
\pysigstartsignatures
\pysigline
{\sphinxbfcode{\sphinxupquote{class\DUrole{w}{ }}}\sphinxcode{\sphinxupquote{server.}}\sphinxbfcode{\sphinxupquote{FileTransferServer}}}
\pysigstopsignatures
\sphinxAtStartPar
Bases: \sphinxcode{\sphinxupquote{object}}
\index{receive\_file() (server.FileTransferServer method)@\spxentry{receive\_file()}\spxextra{server.FileTransferServer method}}

\begin{fulllineitems}
\phantomsection\label{\detokenize{index:server.FileTransferServer.receive_file}}
\pysigstartsignatures
\pysiglinewithargsret
{\sphinxbfcode{\sphinxupquote{receive\_file}}}
{}
{}
\pysigstopsignatures
\sphinxAtStartPar
receiver function for a file transfer destination/server saves data to a file destination (constant)
\begin{quote}\begin{description}
\sphinxlineitem{Parameters}
\sphinxAtStartPar
\sphinxstyleliteralstrong{\sphinxupquote{self}} ({\hyperref[\detokenize{index:server.FileTransferServer}]{\sphinxcrossref{\sphinxstyleliteralemphasis{\sphinxupquote{FileTransferServer}}}}}) \textendash{} the server object

\end{description}\end{quote}

\end{fulllineitems}


\end{fulllineitems}




\renewcommand{\indexname}{Index}
\printindex
\end{document}