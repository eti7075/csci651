%% Generated by Sphinx.
\def\sphinxdocclass{report}
\documentclass[letterpaper,10pt,english,openany,oneside]{sphinxmanual}
\ifdefined\pdfpxdimen
   \let\sphinxpxdimen\pdfpxdimen\else\newdimen\sphinxpxdimen
\fi \sphinxpxdimen=.75bp\relax
\ifdefined\pdfimageresolution
    \pdfimageresolution= \numexpr \dimexpr1in\relax/\sphinxpxdimen\relax
\fi
%% let collapsible pdf bookmarks panel have high depth per default
\PassOptionsToPackage{bookmarksdepth=5}{hyperref}

\PassOptionsToPackage{booktabs}{sphinx}
\PassOptionsToPackage{colorrows}{sphinx}

\PassOptionsToPackage{warn}{textcomp}
\usepackage[utf8]{inputenc}
\ifdefined\DeclareUnicodeCharacter
% support both utf8 and utf8x syntaxes
  \ifdefined\DeclareUnicodeCharacterAsOptional
    \def\sphinxDUC#1{\DeclareUnicodeCharacter{"#1}}
  \else
    \let\sphinxDUC\DeclareUnicodeCharacter
  \fi
  \sphinxDUC{00A0}{\nobreakspace}
  \sphinxDUC{2500}{\sphinxunichar{2500}}
  \sphinxDUC{2502}{\sphinxunichar{2502}}
  \sphinxDUC{2514}{\sphinxunichar{2514}}
  \sphinxDUC{251C}{\sphinxunichar{251C}}
  \sphinxDUC{2572}{\textbackslash}
\fi
\usepackage{cmap}
\usepackage[T1]{fontenc}
\usepackage{amsmath,amssymb,amstext}
\usepackage{babel}



\usepackage{tgtermes}
\usepackage{tgheros}
\renewcommand{\ttdefault}{txtt}



\usepackage[Bjarne]{fncychap}
\usepackage{sphinx}

\fvset{fontsize=auto}
\usepackage{geometry}


% Include hyperref last.
\usepackage{hyperref}
% Fix anchor placement for figures with captions.
\usepackage{hypcap}% it must be loaded after hyperref.
% Set up styles of URL: it should be placed after hyperref.
\urlstyle{same}

\addto\captionsenglish{\renewcommand{\contentsname}{Contents:}}

\usepackage{sphinxmessages}
\setcounter{tocdepth}{1}



\title{PktSniffer}
\date{Jan 30, 2025}
\release{0.1}
\author{Ethan Iannicelli}
\newcommand{\sphinxlogo}{\vbox{}}
\renewcommand{\releasename}{Release}
\makeindex
\begin{document}

\ifdefined\shorthandoff
  \ifnum\catcode`\=\string=\active\shorthandoff{=}\fi
  \ifnum\catcode`\"=\active\shorthandoff{"}\fi
\fi

\pagestyle{empty}
\sphinxmaketitle
\pagestyle{plain}
\sphinxtableofcontents
\pagestyle{normal}
\phantomsection\label{\detokenize{index::doc}}



\chapter{Packet Summary}
\label{\detokenize{index:packet-summary}}
\sphinxAtStartPar
To get the summary of the ethernet header, you can
use the \sphinxcode{\sphinxupquote{get\_eth\_summary()}} function:
\index{get\_eth\_summary() (in module pktsniffer)@\spxentry{get\_eth\_summary()}\spxextra{in module pktsniffer}}

\begin{fulllineitems}
\phantomsection\label{\detokenize{index:pktsniffer.get_eth_summary}}
\pysigstartsignatures
\pysiglinewithargsret
{\sphinxcode{\sphinxupquote{pktsniffer.}}\sphinxbfcode{\sphinxupquote{get\_eth\_summary}}}
{\sphinxparam{\DUrole{n}{packet}}}
{}
\pysigstopsignatures
\sphinxAtStartPar
Print selected ethernet header properties in a formatted manner
\begin{quote}\begin{description}
\sphinxlineitem{Parameters}
\sphinxAtStartPar
\sphinxstyleliteralstrong{\sphinxupquote{packet}} (\sphinxstyleliteralemphasis{\sphinxupquote{pyshark packet}}) \textendash{} required packet to be summarized

\end{description}\end{quote}

\end{fulllineitems}


\sphinxAtStartPar
To get the summary of the ip header, you can
use the \sphinxcode{\sphinxupquote{get\_ip\_summary()}} function:
\index{get\_ip\_summary() (in module pktsniffer)@\spxentry{get\_ip\_summary()}\spxextra{in module pktsniffer}}

\begin{fulllineitems}
\phantomsection\label{\detokenize{index:pktsniffer.get_ip_summary}}
\pysigstartsignatures
\pysiglinewithargsret
{\sphinxcode{\sphinxupquote{pktsniffer.}}\sphinxbfcode{\sphinxupquote{get\_ip\_summary}}}
{\sphinxparam{\DUrole{n}{packet}}}
{}
\pysigstopsignatures
\sphinxAtStartPar
Print selected ip header properties in a formatted manner
\begin{quote}\begin{description}
\sphinxlineitem{Parameters}
\sphinxAtStartPar
\sphinxstyleliteralstrong{\sphinxupquote{packet}} (\sphinxstyleliteralemphasis{\sphinxupquote{pyshark packet}}) \textendash{} required packet to be summarized

\end{description}\end{quote}

\end{fulllineitems}


\sphinxAtStartPar
To get the summary of an encapsulated packet, you can
use the \sphinxcode{\sphinxupquote{get\_encapsulated\_packets\_summary()}} function:
\index{get\_encapsulated\_packets\_summary() (in module pktsniffer)@\spxentry{get\_encapsulated\_packets\_summary()}\spxextra{in module pktsniffer}}

\begin{fulllineitems}
\phantomsection\label{\detokenize{index:pktsniffer.get_encapsulated_packets_summary}}
\pysigstartsignatures
\pysiglinewithargsret
{\sphinxcode{\sphinxupquote{pktsniffer.}}\sphinxbfcode{\sphinxupquote{get\_encapsulated\_packets\_summary}}}
{\sphinxparam{\DUrole{n}{packet}}}
{}
\pysigstopsignatures
\sphinxAtStartPar
Print any encapsulated packet(s) in a given packet (not specially
formatted, does not extract specific properties)
\begin{quote}\begin{description}
\sphinxlineitem{Parameters}
\sphinxAtStartPar
\sphinxstyleliteralstrong{\sphinxupquote{packet}} (\sphinxstyleliteralemphasis{\sphinxupquote{pyshark packet}}) \textendash{} required packet to be summarized

\end{description}\end{quote}

\end{fulllineitems}


\sphinxAtStartPar
To get all available packet summaries, you can
use the \sphinxcode{\sphinxupquote{get\_packet\_summary()}} function:
\index{get\_packet\_summary() (in module pktsniffer)@\spxentry{get\_packet\_summary()}\spxextra{in module pktsniffer}}

\begin{fulllineitems}
\phantomsection\label{\detokenize{index:pktsniffer.get_packet_summary}}
\pysigstartsignatures
\pysiglinewithargsret
{\sphinxcode{\sphinxupquote{pktsniffer.}}\sphinxbfcode{\sphinxupquote{get\_packet\_summary}}}
{\sphinxparam{\DUrole{n}{packet}}}
{}
\pysigstopsignatures
\sphinxAtStartPar
Print all available header summaries for a given packet
\begin{quote}\begin{description}
\sphinxlineitem{Parameters}
\sphinxAtStartPar
\sphinxstyleliteralstrong{\sphinxupquote{packet}} (\sphinxstyleliteralemphasis{\sphinxupquote{pyshark packet}}) \textendash{} required packet to be summarized

\end{description}\end{quote}

\end{fulllineitems}


\sphinxAtStartPar
To filter a list of packets by a host address, you can
use the \sphinxcode{\sphinxupquote{filter\_by\_host()}} function:
\index{filter\_by\_host() (in module pktsniffer)@\spxentry{filter\_by\_host()}\spxextra{in module pktsniffer}}

\begin{fulllineitems}
\phantomsection\label{\detokenize{index:pktsniffer.filter_by_host}}
\pysigstartsignatures
\pysiglinewithargsret
{\sphinxcode{\sphinxupquote{pktsniffer.}}\sphinxbfcode{\sphinxupquote{filter\_by\_host}}}
{\sphinxparam{\DUrole{n}{packets}}\sphinxparamcomma \sphinxparam{\DUrole{n}{host}}}
{}
\pysigstopsignatures
\sphinxAtStartPar
Filter all packets if they contain the host address in either the
packet ip source property or the packet destination property
\begin{quote}\begin{description}
\sphinxlineitem{Parameters}\begin{itemize}
\item {} 
\sphinxAtStartPar
\sphinxstyleliteralstrong{\sphinxupquote{packets}} (\sphinxstyleliteralemphasis{\sphinxupquote{list}}\sphinxstyleliteralemphasis{\sphinxupquote{{[}}}\sphinxstyleliteralemphasis{\sphinxupquote{pyshark packet}}\sphinxstyleliteralemphasis{\sphinxupquote{{]}}}) \textendash{} List of packets

\item {} 
\sphinxAtStartPar
\sphinxstyleliteralstrong{\sphinxupquote{host}} (\sphinxstyleliteralemphasis{\sphinxupquote{MAC address}}) \textendash{} host to filter by

\end{itemize}

\sphinxlineitem{Returns}
\sphinxAtStartPar
the filtered list

\sphinxlineitem{Return type}
\sphinxAtStartPar
list{[}pyshark packet{]}

\end{description}\end{quote}

\end{fulllineitems}


\sphinxAtStartPar
To filter a list of packets by a port, you can
use the \sphinxcode{\sphinxupquote{filter\_by\_port()}} function:
\index{filter\_by\_port() (in module pktsniffer)@\spxentry{filter\_by\_port()}\spxextra{in module pktsniffer}}

\begin{fulllineitems}
\phantomsection\label{\detokenize{index:pktsniffer.filter_by_port}}
\pysigstartsignatures
\pysiglinewithargsret
{\sphinxcode{\sphinxupquote{pktsniffer.}}\sphinxbfcode{\sphinxupquote{filter\_by\_port}}}
{\sphinxparam{\DUrole{n}{packets}}\sphinxparamcomma \sphinxparam{\DUrole{n}{port}}}
{}
\pysigstopsignatures
\sphinxAtStartPar
Filter all packets if they contain the port in either the
encapsulated packet source property or encapsulated packet
destination property
\begin{quote}\begin{description}
\sphinxlineitem{Parameters}\begin{itemize}
\item {} 
\sphinxAtStartPar
\sphinxstyleliteralstrong{\sphinxupquote{packets}} (\sphinxstyleliteralemphasis{\sphinxupquote{list}}\sphinxstyleliteralemphasis{\sphinxupquote{{[}}}\sphinxstyleliteralemphasis{\sphinxupquote{pyshark packet}}\sphinxstyleliteralemphasis{\sphinxupquote{{]}}}) \textendash{} List of packets

\item {} 
\sphinxAtStartPar
\sphinxstyleliteralstrong{\sphinxupquote{port}} \textendash{} port to filter by

\end{itemize}

\sphinxlineitem{Returns}
\sphinxAtStartPar
the filtered list

\sphinxlineitem{Return type}
\sphinxAtStartPar
list{[}pyshark packet{]}

\end{description}\end{quote}

\end{fulllineitems}


\sphinxAtStartPar
To check if a packet has a certain port number, you can
use the \sphinxcode{\sphinxupquote{has\_port()}} function:
\index{has\_port() (in module pktsniffer)@\spxentry{has\_port()}\spxextra{in module pktsniffer}}

\begin{fulllineitems}
\phantomsection\label{\detokenize{index:pktsniffer.has_port}}
\pysigstartsignatures
\pysiglinewithargsret
{\sphinxcode{\sphinxupquote{pktsniffer.}}\sphinxbfcode{\sphinxupquote{has\_port}}}
{\sphinxparam{\DUrole{n}{packet}}\sphinxparamcomma \sphinxparam{\DUrole{n}{port}}}
{}
\pysigstopsignatures
\sphinxAtStartPar
Check if a packet has a port number in a encapsulated TCP
or UDP packet at the source or destination property
\begin{quote}\begin{description}
\sphinxlineitem{Parameters}\begin{itemize}
\item {} 
\sphinxAtStartPar
\sphinxstyleliteralstrong{\sphinxupquote{packet}} (\sphinxstyleliteralemphasis{\sphinxupquote{pyshark packet}}) \textendash{} the packet to be checked

\item {} 
\sphinxAtStartPar
\sphinxstyleliteralstrong{\sphinxupquote{port}} (\sphinxstyleliteralemphasis{\sphinxupquote{Int}}) \textendash{} the port number

\end{itemize}

\sphinxlineitem{Returns}
\sphinxAtStartPar
the boolean value indicating if the packet has the port

\sphinxlineitem{Return type}
\sphinxAtStartPar
boolean

\end{description}\end{quote}

\end{fulllineitems}


\sphinxAtStartPar
To filter a list of packets by a ip version, you can
use the \sphinxcode{\sphinxupquote{filter\_by\_ip()}} function:
\index{filter\_by\_ip() (in module pktsniffer)@\spxentry{filter\_by\_ip()}\spxextra{in module pktsniffer}}

\begin{fulllineitems}
\phantomsection\label{\detokenize{index:pktsniffer.filter_by_ip}}
\pysigstartsignatures
\pysiglinewithargsret
{\sphinxcode{\sphinxupquote{pktsniffer.}}\sphinxbfcode{\sphinxupquote{filter\_by\_ip}}}
{\sphinxparam{\DUrole{n}{packets}}\sphinxparamcomma \sphinxparam{\DUrole{n}{ip}}}
{}
\pysigstopsignatures
\sphinxAtStartPar
Filter all packets if they contain the ip version in the
packet ip header
\begin{quote}\begin{description}
\sphinxlineitem{Parameters}\begin{itemize}
\item {} 
\sphinxAtStartPar
\sphinxstyleliteralstrong{\sphinxupquote{packets}} (\sphinxstyleliteralemphasis{\sphinxupquote{list}}\sphinxstyleliteralemphasis{\sphinxupquote{{[}}}\sphinxstyleliteralemphasis{\sphinxupquote{pyshark packet}}\sphinxstyleliteralemphasis{\sphinxupquote{{]}}}) \textendash{} List of packets

\item {} 
\sphinxAtStartPar
\sphinxstyleliteralstrong{\sphinxupquote{ip}} (\sphinxstyleliteralemphasis{\sphinxupquote{Int}}) \textendash{} ip version to filter by

\end{itemize}

\sphinxlineitem{Returns}
\sphinxAtStartPar
the filtered list

\sphinxlineitem{Return type}
\sphinxAtStartPar
list{[}pyshark packet{]}

\end{description}\end{quote}

\end{fulllineitems}


\sphinxAtStartPar
To filter a list of packets by a net, you can
use the \sphinxcode{\sphinxupquote{filter\_by\_net()}} function:
\index{filter\_by\_net() (in module pktsniffer)@\spxentry{filter\_by\_net()}\spxextra{in module pktsniffer}}

\begin{fulllineitems}
\phantomsection\label{\detokenize{index:pktsniffer.filter_by_net}}
\pysigstartsignatures
\pysiglinewithargsret
{\sphinxcode{\sphinxupquote{pktsniffer.}}\sphinxbfcode{\sphinxupquote{filter\_by\_net}}}
{\sphinxparam{\DUrole{n}{packets}}\sphinxparamcomma \sphinxparam{\DUrole{n}{net}}}
{}
\pysigstopsignatures
\sphinxAtStartPar
Filter all packets if they contain an encapsulated icmp packet
\begin{quote}\begin{description}
\sphinxlineitem{Parameters}
\sphinxAtStartPar
\sphinxstyleliteralstrong{\sphinxupquote{packets}} (\sphinxstyleliteralemphasis{\sphinxupquote{list}}\sphinxstyleliteralemphasis{\sphinxupquote{{[}}}\sphinxstyleliteralemphasis{\sphinxupquote{pyshark packet}}\sphinxstyleliteralemphasis{\sphinxupquote{{]}}}) \textendash{} List of packets

\sphinxlineitem{Returns}
\sphinxAtStartPar
the filtered list

\sphinxlineitem{Return type}
\sphinxAtStartPar
list{[}pyshark packet{]}

\end{description}\end{quote}

\end{fulllineitems}


\sphinxAtStartPar
To filter a list of packets by a tcp, you can
use the \sphinxcode{\sphinxupquote{filter\_by\_tcp()}} function:
\index{filter\_by\_tcp() (in module pktsniffer)@\spxentry{filter\_by\_tcp()}\spxextra{in module pktsniffer}}

\begin{fulllineitems}
\phantomsection\label{\detokenize{index:pktsniffer.filter_by_tcp}}
\pysigstartsignatures
\pysiglinewithargsret
{\sphinxcode{\sphinxupquote{pktsniffer.}}\sphinxbfcode{\sphinxupquote{filter\_by\_tcp}}}
{\sphinxparam{\DUrole{n}{packets}}}
{}
\pysigstopsignatures
\sphinxAtStartPar
Filter all packets if they contain the same address in
either the packet ip source or destination property
\begin{quote}\begin{description}
\sphinxlineitem{Parameters}\begin{itemize}
\item {} 
\sphinxAtStartPar
\sphinxstyleliteralstrong{\sphinxupquote{packets}} (\sphinxstyleliteralemphasis{\sphinxupquote{list}}\sphinxstyleliteralemphasis{\sphinxupquote{{[}}}\sphinxstyleliteralemphasis{\sphinxupquote{pyshark packet}}\sphinxstyleliteralemphasis{\sphinxupquote{{]}}}) \textendash{} List of packets

\item {} 
\sphinxAtStartPar
\sphinxstyleliteralstrong{\sphinxupquote{net}} (\sphinxstyleliteralemphasis{\sphinxupquote{MAC address}}) \textendash{} net to filter by

\end{itemize}

\sphinxlineitem{Returns}
\sphinxAtStartPar
the filtered list

\sphinxlineitem{Return type}
\sphinxAtStartPar
list{[}pyshark packet{]}

\end{description}\end{quote}

\end{fulllineitems}


\sphinxAtStartPar
To filter a list of packets by a udp, you can
use the \sphinxcode{\sphinxupquote{filter\_by\_udp()}} function:
\index{filter\_by\_udp() (in module pktsniffer)@\spxentry{filter\_by\_udp()}\spxextra{in module pktsniffer}}

\begin{fulllineitems}
\phantomsection\label{\detokenize{index:pktsniffer.filter_by_udp}}
\pysigstartsignatures
\pysiglinewithargsret
{\sphinxcode{\sphinxupquote{pktsniffer.}}\sphinxbfcode{\sphinxupquote{filter\_by\_udp}}}
{\sphinxparam{\DUrole{n}{packets}}}
{}
\pysigstopsignatures
\sphinxAtStartPar
Filter all packets if they contain an encapsulated tcp packet
\begin{quote}\begin{description}
\sphinxlineitem{Parameters}
\sphinxAtStartPar
\sphinxstyleliteralstrong{\sphinxupquote{packets}} (\sphinxstyleliteralemphasis{\sphinxupquote{list}}\sphinxstyleliteralemphasis{\sphinxupquote{{[}}}\sphinxstyleliteralemphasis{\sphinxupquote{pyshark packet}}\sphinxstyleliteralemphasis{\sphinxupquote{{]}}}) \textendash{} List of packets

\sphinxlineitem{Returns}
\sphinxAtStartPar
the filtered list

\sphinxlineitem{Return type}
\sphinxAtStartPar
list{[}pyshark packet{]}

\end{description}\end{quote}

\end{fulllineitems}


\sphinxAtStartPar
To filter a list of packets by a icmp, you can
use the \sphinxcode{\sphinxupquote{filter\_by\_icmp()}} function:
\index{filter\_by\_icmp() (in module pktsniffer)@\spxentry{filter\_by\_icmp()}\spxextra{in module pktsniffer}}

\begin{fulllineitems}
\phantomsection\label{\detokenize{index:pktsniffer.filter_by_icmp}}
\pysigstartsignatures
\pysiglinewithargsret
{\sphinxcode{\sphinxupquote{pktsniffer.}}\sphinxbfcode{\sphinxupquote{filter\_by\_icmp}}}
{\sphinxparam{\DUrole{n}{packets}}}
{}
\pysigstopsignatures
\sphinxAtStartPar
Filter all packets if they contain an encapsulated udp packet
\begin{quote}\begin{description}
\sphinxlineitem{Parameters}
\sphinxAtStartPar
\sphinxstyleliteralstrong{\sphinxupquote{packets}} (\sphinxstyleliteralemphasis{\sphinxupquote{list}}\sphinxstyleliteralemphasis{\sphinxupquote{{[}}}\sphinxstyleliteralemphasis{\sphinxupquote{pyshark packet}}\sphinxstyleliteralemphasis{\sphinxupquote{{]}}}) \textendash{} List of packets

\sphinxlineitem{Returns}
\sphinxAtStartPar
the filtered list

\sphinxlineitem{Return type}
\sphinxAtStartPar
list{[}pyshark packet{]}

\end{description}\end{quote}

\end{fulllineitems}


\sphinxAtStartPar
To filter a list of packets by all filters, use
the \sphinxcode{\sphinxupquote{filter\_packets()}} function:
\index{filter\_packets() (in module pktsniffer)@\spxentry{filter\_packets()}\spxextra{in module pktsniffer}}

\begin{fulllineitems}
\phantomsection\label{\detokenize{index:pktsniffer.filter_packets}}
\pysigstartsignatures
\pysiglinewithargsret
{\sphinxcode{\sphinxupquote{pktsniffer.}}\sphinxbfcode{\sphinxupquote{filter\_packets}}}
{\sphinxparam{\DUrole{n}{packets}}\sphinxparamcomma \sphinxparam{\DUrole{n}{filters}}}
{}
\pysigstopsignatures
\sphinxAtStartPar
This function uses all the filtering helper functions to filter
a list of packets given a set of (active) filters
\begin{quote}\begin{description}
\sphinxlineitem{Parameters}\begin{itemize}
\item {} 
\sphinxAtStartPar
\sphinxstyleliteralstrong{\sphinxupquote{packets}} (\sphinxstyleliteralemphasis{\sphinxupquote{list}}\sphinxstyleliteralemphasis{\sphinxupquote{{[}}}\sphinxstyleliteralemphasis{\sphinxupquote{pyshark packet}}\sphinxstyleliteralemphasis{\sphinxupquote{{]}}}) \textendash{} list of packets

\item {} 
\sphinxAtStartPar
\sphinxstyleliteralstrong{\sphinxupquote{filters}} (\sphinxstyleliteralemphasis{\sphinxupquote{map\textless{}string}}\sphinxstyleliteralemphasis{\sphinxupquote{, }}\sphinxstyleliteralemphasis{\sphinxupquote{value\textgreater{}}}) \textendash{} the filters to use in filtering the packets

\end{itemize}

\sphinxlineitem{Returns}
\sphinxAtStartPar
list of filtered packets

\sphinxlineitem{Return type}
\sphinxAtStartPar
list{[}pyshark packet{]}

\end{description}\end{quote}

\end{fulllineitems}


\sphinxAtStartPar
To initiate the program parser, you can use
the \sphinxcode{\sphinxupquote{initialize\_parser()}} function:
\index{initialize\_parser() (in module pktsniffer)@\spxentry{initialize\_parser()}\spxextra{in module pktsniffer}}

\begin{fulllineitems}
\phantomsection\label{\detokenize{index:pktsniffer.initialize_parser}}
\pysigstartsignatures
\pysiglinewithargsret
{\sphinxcode{\sphinxupquote{pktsniffer.}}\sphinxbfcode{\sphinxupquote{initialize\_parser}}}
{}
{}
\pysigstopsignatures
\sphinxAtStartPar
This function creates and defines the parser for the packet
sniffer program, including file arguments, filtering arguments,
and count arguments
\begin{quote}\begin{description}
\sphinxlineitem{Returns}
\sphinxAtStartPar
the initialized parser

\sphinxlineitem{Return type}
\sphinxAtStartPar
ArgParser

\end{description}\end{quote}

\end{fulllineitems}




\renewcommand{\indexname}{Index}
\printindex
\end{document}