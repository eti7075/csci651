%% Generated by Sphinx.
\def\sphinxdocclass{report}
\documentclass[letterpaper,10pt,english,openany,oneside]{sphinxmanual}
\ifdefined\pdfpxdimen
   \let\sphinxpxdimen\pdfpxdimen\else\newdimen\sphinxpxdimen
\fi \sphinxpxdimen=.75bp\relax
\ifdefined\pdfimageresolution
    \pdfimageresolution= \numexpr \dimexpr1in\relax/\sphinxpxdimen\relax
\fi
%% let collapsible pdf bookmarks panel have high depth per default
\PassOptionsToPackage{bookmarksdepth=5}{hyperref}

\PassOptionsToPackage{booktabs}{sphinx}
\PassOptionsToPackage{colorrows}{sphinx}

\PassOptionsToPackage{warn}{textcomp}
\usepackage[utf8]{inputenc}
\ifdefined\DeclareUnicodeCharacter
% support both utf8 and utf8x syntaxes
  \ifdefined\DeclareUnicodeCharacterAsOptional
    \def\sphinxDUC#1{\DeclareUnicodeCharacter{"#1}}
  \else
    \let\sphinxDUC\DeclareUnicodeCharacter
  \fi
  \sphinxDUC{00A0}{\nobreakspace}
  \sphinxDUC{2500}{\sphinxunichar{2500}}
  \sphinxDUC{2502}{\sphinxunichar{2502}}
  \sphinxDUC{2514}{\sphinxunichar{2514}}
  \sphinxDUC{251C}{\sphinxunichar{251C}}
  \sphinxDUC{2572}{\textbackslash}
\fi
\usepackage{cmap}
\usepackage[T1]{fontenc}
\usepackage{amsmath,amssymb,amstext}
\usepackage{babel}



\usepackage{tgtermes}
\usepackage{tgheros}
\renewcommand{\ttdefault}{txtt}



\usepackage[Bjarne]{fncychap}
\usepackage{sphinx}

\fvset{fontsize=auto}
\usepackage{geometry}


% Include hyperref last.
\usepackage{hyperref}
% Fix anchor placement for figures with captions.
\usepackage{hypcap}% it must be loaded after hyperref.
% Set up styles of URL: it should be placed after hyperref.
\urlstyle{same}

\addto\captionsenglish{\renewcommand{\contentsname}{Contents:}}

\usepackage{sphinxmessages}
\setcounter{tocdepth}{1}



\title{Peer 2 Peer File Transfer}
\date{Apr 11, 2025}
\release{1.0}
\author{Ethan Iannicelli}
\newcommand{\sphinxlogo}{\vbox{}}
\renewcommand{\releasename}{Release}
\makeindex
\begin{document}

\ifdefined\shorthandoff
  \ifnum\catcode`\=\string=\active\shorthandoff{=}\fi
  \ifnum\catcode`\"=\active\shorthandoff{"}\fi
\fi

\pagestyle{empty}
\sphinxmaketitle
\pagestyle{plain}
\sphinxtableofcontents
\pagestyle{normal}
\phantomsection\label{\detokenize{index::doc}}



\chapter{Peer}
\label{\detokenize{index:peer}}
\sphinxAtStartPar
The \sphinxcode{\sphinxupquote{Peer}} class represents a Peer entity in a P2P file transfer
system that supports broadcast discovery and chunk file shring with
other peers in the network:
\index{Peer (class in peer)@\spxentry{Peer}\spxextra{class in peer}}

\begin{fulllineitems}
\phantomsection\label{\detokenize{index:peer.Peer}}
\pysigstartsignatures
\pysiglinewithargsret
{\sphinxbfcode{\sphinxupquote{class\DUrole{w}{ }}}\sphinxcode{\sphinxupquote{peer.}}\sphinxbfcode{\sphinxupquote{Peer}}}
{\sphinxparam{\DUrole{n}{discovery\_port}}\sphinxparamcomma \sphinxparam{\DUrole{n}{transfer\_port}}}
{}
\pysigstopsignatures
\sphinxAtStartPar
Bases: \sphinxcode{\sphinxupquote{object}}
\index{start() (peer.Peer method)@\spxentry{start()}\spxextra{peer.Peer method}}

\begin{fulllineitems}
\phantomsection\label{\detokenize{index:peer.Peer.start}}
\pysigstartsignatures
\pysiglinewithargsret
{\sphinxbfcode{\sphinxupquote{start}}}
{}
{}
\pysigstopsignatures
\sphinxAtStartPar
Start peer broadcasting and transfer services. Also start command line input loop
for handling commands.
\sphinxhyphen{} list: output the files and chunks available on this peer
\sphinxhyphen{} download \textless{}file\textgreater{}: requests all available peers for any chunks they have for \textless{}file\textgreater{}.
use threading for each request, and accumulate in a shared local memory to build the
file on the receiver end.
\sphinxhyphen{} exit: stop the peer and shut down

\end{fulllineitems}


\end{fulllineitems}



\chapter{Discovery}
\label{\detokenize{index:discovery}}
\sphinxAtStartPar
The \sphinxcode{\sphinxupquote{PeerDiscovery}} class represents the entity responsible for broadcasting
and discovering other peers in a network. This broadcast is ‘parent’ peer, and
detects other discovery entities that are serving their respective ‘parent’ peers:
\index{PeerDiscovery (class in discovery)@\spxentry{PeerDiscovery}\spxextra{class in discovery}}

\begin{fulllineitems}
\phantomsection\label{\detokenize{index:discovery.PeerDiscovery}}
\pysigstartsignatures
\pysiglinewithargsret
{\sphinxbfcode{\sphinxupquote{class\DUrole{w}{ }}}\sphinxcode{\sphinxupquote{discovery.}}\sphinxbfcode{\sphinxupquote{PeerDiscovery}}}
{\sphinxparam{\DUrole{n}{transfer\_port}}\sphinxparamcomma \sphinxparam{\DUrole{n}{port}}}
{}
\pysigstopsignatures
\sphinxAtStartPar
Bases: \sphinxcode{\sphinxupquote{object}}
\index{broadcast\_announcement() (discovery.PeerDiscovery method)@\spxentry{broadcast\_announcement()}\spxextra{discovery.PeerDiscovery method}}

\begin{fulllineitems}
\phantomsection\label{\detokenize{index:discovery.PeerDiscovery.broadcast_announcement}}
\pysigstartsignatures
\pysiglinewithargsret
{\sphinxbfcode{\sphinxupquote{broadcast\_announcement}}}
{}
{}
\pysigstopsignatures
\sphinxAtStartPar
Broadcast availability to the network on the broadcast port. send the sender port
the distributee port with this broadcast, as well as the ‘start’ itentifier

\end{fulllineitems}

\index{listen\_for\_peers() (discovery.PeerDiscovery method)@\spxentry{listen\_for\_peers()}\spxextra{discovery.PeerDiscovery method}}

\begin{fulllineitems}
\phantomsection\label{\detokenize{index:discovery.PeerDiscovery.listen_for_peers}}
\pysigstartsignatures
\pysiglinewithargsret
{\sphinxbfcode{\sphinxupquote{listen\_for\_peers}}}
{}
{}
\pysigstopsignatures
\sphinxAtStartPar
Listen for incoming peer announcements on the broadcast port. when it receives a
broadcast, extract the peer ports that are packaged in the broadcast. if the broadcast
is defined as ‘start’, add to known peers \sphinxhyphen{} if ‘stop’ remove from known peers

\end{fulllineitems}

\index{start() (discovery.PeerDiscovery method)@\spxentry{start()}\spxextra{discovery.PeerDiscovery method}}

\begin{fulllineitems}
\phantomsection\label{\detokenize{index:discovery.PeerDiscovery.start}}
\pysigstartsignatures
\pysiglinewithargsret
{\sphinxbfcode{\sphinxupquote{start}}}
{}
{}
\pysigstopsignatures
\sphinxAtStartPar
listening for peers and broadcasting itself

\end{fulllineitems}

\index{stop() (discovery.PeerDiscovery method)@\spxentry{stop()}\spxextra{discovery.PeerDiscovery method}}

\begin{fulllineitems}
\phantomsection\label{\detokenize{index:discovery.PeerDiscovery.stop}}
\pysigstartsignatures
\pysiglinewithargsret
{\sphinxbfcode{\sphinxupquote{stop}}}
{}
{}
\pysigstopsignatures
\sphinxAtStartPar
shut down this discovery entity. broadcast itself with ‘stop’ flag, and set running to false

\end{fulllineitems}


\end{fulllineitems}



\chapter{Transfer}
\label{\detokenize{index:transfer}}
\sphinxAtStartPar
The \sphinxcode{\sphinxupquote{FileTransfer}} class represents the entity responsible for distributing, receiving,
requesting, and sending file chunks to other peers. It is always in a state of distributing
and receiving file chunks from other peers, and when it’s peer starts a file request it is
responsible for getting this data from other peers and saving it to a file:
\index{FileTransfer (class in transfer)@\spxentry{FileTransfer}\spxextra{class in transfer}}

\begin{fulllineitems}
\phantomsection\label{\detokenize{index:transfer.FileTransfer}}
\pysigstartsignatures
\pysiglinewithargsret
{\sphinxbfcode{\sphinxupquote{class\DUrole{w}{ }}}\sphinxcode{\sphinxupquote{transfer.}}\sphinxbfcode{\sphinxupquote{FileTransfer}}}
{\sphinxparam{\DUrole{n}{port}}\sphinxparamcomma \sphinxparam{\DUrole{n}{discovery}}\sphinxparamcomma \sphinxparam{\DUrole{n}{files}}}
{}
\pysigstopsignatures
\sphinxAtStartPar
Bases: \sphinxcode{\sphinxupquote{object}}
\index{distribute\_files() (transfer.FileTransfer method)@\spxentry{distribute\_files()}\spxextra{transfer.FileTransfer method}}

\begin{fulllineitems}
\phantomsection\label{\detokenize{index:transfer.FileTransfer.distribute_files}}
\pysigstartsignatures
\pysiglinewithargsret
{\sphinxbfcode{\sphinxupquote{distribute\_files}}}
{}
{}
\pysigstopsignatures
\sphinxAtStartPar
periodically (every 10 seconds) randomly send chunks of files that this peer knows to other peers.
all chunks are sent, but the peer is randomly chosen. this is done to simulate peers not having perfect knowledge
of the files from the start and to simulate rate limits of data sent.

\end{fulllineitems}

\index{handle\_client() (transfer.FileTransfer method)@\spxentry{handle\_client()}\spxextra{transfer.FileTransfer method}}

\begin{fulllineitems}
\phantomsection\label{\detokenize{index:transfer.FileTransfer.handle_client}}
\pysigstartsignatures
\pysiglinewithargsret
{\sphinxbfcode{\sphinxupquote{handle\_client}}}
{}
{}
\pysigstopsignatures
\sphinxAtStartPar
Handle an incoming file request. receive a file name to initialize this process, and then send all
chunks that this peer has to the sender address in separate requests.

\end{fulllineitems}

\index{receive\_distributed\_files() (transfer.FileTransfer method)@\spxentry{receive\_distributed\_files()}\spxextra{transfer.FileTransfer method}}

\begin{fulllineitems}
\phantomsection\label{\detokenize{index:transfer.FileTransfer.receive_distributed_files}}
\pysigstartsignatures
\pysiglinewithargsret
{\sphinxbfcode{\sphinxupquote{receive\_distributed\_files}}}
{}
{}
\pysigstopsignatures
\sphinxAtStartPar
recieve chunks of files sent to this port. extract the filename, chunk number, and chunk and save it to the
files dictionary representing local session storage.

\end{fulllineitems}

\index{request\_file() (transfer.FileTransfer method)@\spxentry{request\_file()}\spxextra{transfer.FileTransfer method}}

\begin{fulllineitems}
\phantomsection\label{\detokenize{index:transfer.FileTransfer.request_file}}
\pysigstartsignatures
\pysiglinewithargsret
{\sphinxbfcode{\sphinxupquote{request\_file}}}
{\sphinxparam{\DUrole{n}{filename}}\sphinxparamcomma \sphinxparam{\DUrole{n}{peer\_port}}\sphinxparamcomma \sphinxparam{\DUrole{n}{chunks}}\sphinxparamcomma \sphinxparam{\DUrole{n}{peer}}}
{}
\pysigstopsignatures
\sphinxAtStartPar
Request a file from a peer. This is called as one of many threads, so we use a global chunks dictionary
to store all downloaded chunks. When the chunks dictionary is ‘full’ (all chunks for file are present), this
thread marks as so and attempts to write to a file. If a timeout occurs, the user is notified and an incomplete
file is downloaded and written
\begin{quote}\begin{description}
\sphinxlineitem{Parameters}\begin{itemize}
\item {} 
\sphinxAtStartPar
\sphinxstyleliteralstrong{\sphinxupquote{filename}} (\sphinxstyleliteralemphasis{\sphinxupquote{string}}) \textendash{} the name of the file to be downloaded

\item {} 
\sphinxAtStartPar
\sphinxstyleliteralstrong{\sphinxupquote{peer\_port}} (\sphinxstyleliteralemphasis{\sphinxupquote{string}}) \textendash{} the port of the peer that we want to request from

\item {} 
\sphinxAtStartPar
\sphinxstyleliteralstrong{\sphinxupquote{chunks}} (\sphinxstyleliteralemphasis{\sphinxupquote{map\textless{}string}}\sphinxstyleliteralemphasis{\sphinxupquote{, }}\sphinxstyleliteralemphasis{\sphinxupquote{string\textgreater{}}}) \textendash{} global dictionary for this file’s chunks being stored in

\item {} 
\sphinxAtStartPar
\sphinxstyleliteralstrong{\sphinxupquote{peer}} ({\hyperref[\detokenize{index:peer.Peer}]{\sphinxcrossref{\sphinxstyleliteralemphasis{\sphinxupquote{Peer}}}}}) \textendash{} the peer that called this function, used to know status of other threads that the peer spawned.

\end{itemize}

\end{description}\end{quote}

\end{fulllineitems}

\index{start\_server() (transfer.FileTransfer method)@\spxentry{start\_server()}\spxextra{transfer.FileTransfer method}}

\begin{fulllineitems}
\phantomsection\label{\detokenize{index:transfer.FileTransfer.start_server}}
\pysigstartsignatures
\pysiglinewithargsret
{\sphinxbfcode{\sphinxupquote{start\_server}}}
{}
{}
\pysigstopsignatures
\sphinxAtStartPar
Start the file transfer server. start threads to handle requests from other peers, distribute chunks to other
peers, and receive distributed chunks from other peers

\end{fulllineitems}


\end{fulllineitems}



\chapter{Packet}
\label{\detokenize{index:packet}}
\sphinxAtStartPar
To calculate the psuedo udp checksum of the data, use the \sphinxcode{\sphinxupquote{udp\_checksum()}} function. This
function is neccessary for maintaining packet integrity:
\index{udp\_checksum() (in module packet)@\spxentry{udp\_checksum()}\spxextra{in module packet}}

\begin{fulllineitems}
\phantomsection\label{\detokenize{index:packet.udp_checksum}}
\pysigstartsignatures
\pysiglinewithargsret
{\sphinxcode{\sphinxupquote{packet.}}\sphinxbfcode{\sphinxupquote{udp\_checksum}}}
{\sphinxparam{\DUrole{n}{data}}}
{}
\pysigstopsignatures
\sphinxAtStartPar
perform a psuedo udp checksum by reducing the data to 4 bytes and taking one’s complement
\begin{quote}\begin{description}
\sphinxlineitem{Parameters}
\sphinxAtStartPar
\sphinxstyleliteralstrong{\sphinxupquote{data}} (\sphinxstyleliteralemphasis{\sphinxupquote{bitstring}}) \textendash{} the data that the checksum is created from

\sphinxlineitem{Returns}
\sphinxAtStartPar
generated checksum

\sphinxlineitem{Return type}
\sphinxAtStartPar
int

\end{description}\end{quote}

\end{fulllineitems}


\sphinxAtStartPar
To create a packet from a chunk number and a chunk of data, use the \sphinxcode{\sphinxupquote{create\_packet()}} function:
\index{create\_packet() (in module packet)@\spxentry{create\_packet()}\spxextra{in module packet}}

\begin{fulllineitems}
\phantomsection\label{\detokenize{index:packet.create_packet}}
\pysigstartsignatures
\pysiglinewithargsret
{\sphinxcode{\sphinxupquote{packet.}}\sphinxbfcode{\sphinxupquote{create\_packet}}}
{\sphinxparam{\DUrole{n}{data}}\sphinxparamcomma \sphinxparam{\DUrole{n}{chunk\_num}}}
{}
\pysigstopsignatures
\sphinxAtStartPar
create a packet using packet data and a chunk number
\begin{quote}\begin{description}
\sphinxlineitem{Parameters}\begin{itemize}
\item {} 
\sphinxAtStartPar
\sphinxstyleliteralstrong{\sphinxupquote{data}} (\sphinxstyleliteralemphasis{\sphinxupquote{bitstring}}) \textendash{} the data to be included in the packet

\item {} 
\sphinxAtStartPar
\sphinxstyleliteralstrong{\sphinxupquote{chunk\_num}} (\sphinxstyleliteralemphasis{\sphinxupquote{int}}) \textendash{} the chunk number associated with this data

\end{itemize}

\sphinxlineitem{Returns}
\sphinxAtStartPar
bitstring representing formed packet

\sphinxlineitem{Return type}
\sphinxAtStartPar
bitstring

\end{description}\end{quote}

\end{fulllineitems}


\sphinxAtStartPar
To parse the components of a packet into its different parts, used the \sphinxcode{\sphinxupquote{parse\_packet()}} function.
The parts of the packet returned are the checksum, chunk\_number, and the chunk data:
\index{parse\_packet() (in module packet)@\spxentry{parse\_packet()}\spxextra{in module packet}}

\begin{fulllineitems}
\phantomsection\label{\detokenize{index:packet.parse_packet}}
\pysigstartsignatures
\pysiglinewithargsret
{\sphinxcode{\sphinxupquote{packet.}}\sphinxbfcode{\sphinxupquote{parse\_packet}}}
{\sphinxparam{\DUrole{n}{packet}}}
{}
\pysigstopsignatures
\sphinxAtStartPar
extracts checksum, chunk\_num, and data from a packet
\begin{quote}\begin{description}
\sphinxlineitem{Parameters}
\sphinxAtStartPar
\sphinxstyleliteralstrong{\sphinxupquote{packet}} (\sphinxstyleliteralemphasis{\sphinxupquote{bitstring}}) \textendash{} formatted packet

\sphinxlineitem{Returns}
\sphinxAtStartPar
3 tuple of check, chunk\_num, data

\sphinxlineitem{Return type}
\sphinxAtStartPar
tuple

\end{description}\end{quote}

\end{fulllineitems}



\chapter{Main}
\label{\detokenize{index:main}}
\sphinxAtStartPar
To create and return a parser for the program, use the \sphinxcode{\sphinxupquote{parse\_arguments()}} function:
\index{parse\_arguments() (in module main)@\spxentry{parse\_arguments()}\spxextra{in module main}}

\begin{fulllineitems}
\phantomsection\label{\detokenize{index:main.parse_arguments}}
\pysigstartsignatures
\pysiglinewithargsret
{\sphinxcode{\sphinxupquote{main.}}\sphinxbfcode{\sphinxupquote{parse\_arguments}}}
{}
{}
\pysigstopsignatures
\sphinxAtStartPar
create and return argument parser. the parser handles the broadcasting port and the default transfer port.
the transfer port requires it, +1, +2, +3 ports to be unused before running the program.

\end{fulllineitems}


\sphinxAtStartPar
The main function of this program is \sphinxcode{\sphinxupquote{main()}}. It parsers arguments using a parser, and starts
and instance of the peer:
\index{main() (in module main)@\spxentry{main()}\spxextra{in module main}}

\begin{fulllineitems}
\phantomsection\label{\detokenize{index:main.main}}
\pysigstartsignatures
\pysiglinewithargsret
{\sphinxcode{\sphinxupquote{main.}}\sphinxbfcode{\sphinxupquote{main}}}
{}
{}
\pysigstopsignatures
\sphinxAtStartPar
main function for P2P program, starts Peer and accepts arguments.

\end{fulllineitems}



\chapter{Logger}
\label{\detokenize{index:logger}}
\sphinxAtStartPar
To get the logger from the program, use the \sphinxcode{\sphinxupquote{get\_logger()}} function:
\index{get\_logger() (in module utils.logger)@\spxentry{get\_logger()}\spxextra{in module utils.logger}}

\begin{fulllineitems}
\phantomsection\label{\detokenize{index:utils.logger.get_logger}}
\pysigstartsignatures
\pysiglinewithargsret
{\sphinxcode{\sphinxupquote{utils.logger.}}\sphinxbfcode{\sphinxupquote{get\_logger}}}
{\sphinxparam{\DUrole{n}{name}\DUrole{o}{=}\DUrole{default_value}{\textquotesingle{}P2P\textquotesingle{}}}}
{}
\pysigstopsignatures
\sphinxAtStartPar
creates and returns a configured logger for the UI of the program
\begin{quote}\begin{description}
\sphinxlineitem{Parameters}
\sphinxAtStartPar
\sphinxstyleliteralstrong{\sphinxupquote{name}} (\sphinxstyleliteralemphasis{\sphinxupquote{string}}) \textendash{} name of the program/app

\sphinxlineitem{Returns}
\sphinxAtStartPar
logger

\sphinxlineitem{Return type}
\sphinxAtStartPar
logger

\end{description}\end{quote}

\end{fulllineitems}


\sphinxAtStartPar
To create a formatted string output for a in memory stored file dictionary containing chunks, use
the \sphinxcode{\sphinxupquote{format\_file\_chunks()}} function:
\index{format\_file\_chunks() (in module utils.logger)@\spxentry{format\_file\_chunks()}\spxextra{in module utils.logger}}

\begin{fulllineitems}
\phantomsection\label{\detokenize{index:utils.logger.format_file_chunks}}
\pysigstartsignatures
\pysiglinewithargsret
{\sphinxcode{\sphinxupquote{utils.logger.}}\sphinxbfcode{\sphinxupquote{format\_file\_chunks}}}
{\sphinxparam{\DUrole{n}{files}}}
{}
\pysigstopsignatures
\sphinxAtStartPar
helper function to format the output for the files store in local memory on a peer
this is not really a logger function, but is used in output
\begin{quote}\begin{description}
\sphinxlineitem{Parameters}
\sphinxAtStartPar
\sphinxstyleliteralstrong{\sphinxupquote{files}} (\sphinxstyleliteralemphasis{\sphinxupquote{map\textless{}string}}\sphinxstyleliteralemphasis{\sphinxupquote{, }}\sphinxstyleliteralemphasis{\sphinxupquote{map\textless{}string}}\sphinxstyleliteralemphasis{\sphinxupquote{, }}\sphinxstyleliteralemphasis{\sphinxupquote{string\textgreater{}\textgreater{}}}) \textendash{} files and chunks to output

\sphinxlineitem{Returns}
\sphinxAtStartPar
files and the number of chunks available in string format

\sphinxlineitem{Return type}
\sphinxAtStartPar
string

\end{description}\end{quote}

\end{fulllineitems}




\renewcommand{\indexname}{Index}
\printindex
\end{document}