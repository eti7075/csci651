%% Generated by Sphinx.
\def\sphinxdocclass{report}
\documentclass[letterpaper,10pt,english,openany,oneside]{sphinxmanual}
\ifdefined\pdfpxdimen
   \let\sphinxpxdimen\pdfpxdimen\else\newdimen\sphinxpxdimen
\fi \sphinxpxdimen=.75bp\relax
\ifdefined\pdfimageresolution
    \pdfimageresolution= \numexpr \dimexpr1in\relax/\sphinxpxdimen\relax
\fi
%% let collapsible pdf bookmarks panel have high depth per default
\PassOptionsToPackage{bookmarksdepth=5}{hyperref}

\PassOptionsToPackage{booktabs}{sphinx}
\PassOptionsToPackage{colorrows}{sphinx}

\PassOptionsToPackage{warn}{textcomp}
\usepackage[utf8]{inputenc}
\ifdefined\DeclareUnicodeCharacter
% support both utf8 and utf8x syntaxes
  \ifdefined\DeclareUnicodeCharacterAsOptional
    \def\sphinxDUC#1{\DeclareUnicodeCharacter{"#1}}
  \else
    \let\sphinxDUC\DeclareUnicodeCharacter
  \fi
  \sphinxDUC{00A0}{\nobreakspace}
  \sphinxDUC{2500}{\sphinxunichar{2500}}
  \sphinxDUC{2502}{\sphinxunichar{2502}}
  \sphinxDUC{2514}{\sphinxunichar{2514}}
  \sphinxDUC{251C}{\sphinxunichar{251C}}
  \sphinxDUC{2572}{\textbackslash}
\fi
\usepackage{cmap}
\usepackage[T1]{fontenc}
\usepackage{amsmath,amssymb,amstext}
\usepackage{babel}



\usepackage{tgtermes}
\usepackage{tgheros}
\renewcommand{\ttdefault}{txtt}



\usepackage[Bjarne]{fncychap}
\usepackage{sphinx}

\fvset{fontsize=auto}
\usepackage{geometry}


% Include hyperref last.
\usepackage{hyperref}
% Fix anchor placement for figures with captions.
\usepackage{hypcap}% it must be loaded after hyperref.
% Set up styles of URL: it should be placed after hyperref.
\urlstyle{same}

\addto\captionsenglish{\renewcommand{\contentsname}{Contents:}}

\usepackage{sphinxmessages}
\setcounter{tocdepth}{1}



\title{ping\_and\_traceroute}
\date{Feb 12, 2025}
\release{}
\author{Ethan Iannicelli}
\newcommand{\sphinxlogo}{\vbox{}}
\renewcommand{\releasename}{}
\makeindex
\begin{document}

\ifdefined\shorthandoff
  \ifnum\catcode`\=\string=\active\shorthandoff{=}\fi
  \ifnum\catcode`\"=\active\shorthandoff{"}\fi
\fi

\pagestyle{empty}
\sphinxmaketitle
\pagestyle{plain}
\sphinxtableofcontents
\pagestyle{normal}
\phantomsection\label{\detokenize{index::doc}}



\chapter{Ping Summary}
\label{\detokenize{index:ping-summary}}
\sphinxAtStartPar
To get the checksum of an ICMP packet based on the
string representation of the packet, use the \sphinxcode{\sphinxupquote{checksum()}} function:
\index{checksum() (in module my\_ping)@\spxentry{checksum()}\spxextra{in module my\_ping}}

\begin{fulllineitems}
\phantomsection\label{\detokenize{index:my_ping.checksum}}
\pysigstartsignatures
\pysiglinewithargsret
{\sphinxcode{\sphinxupquote{my\_ping.}}\sphinxbfcode{\sphinxupquote{checksum}}}
{\sphinxparam{\DUrole{n}{data}}}
{}
\pysigstopsignatures
\sphinxAtStartPar
Creates the checksum for a given data for icmp packet
\begin{quote}\begin{description}
\sphinxlineitem{Parameters}
\sphinxAtStartPar
\sphinxstyleliteralstrong{\sphinxupquote{data}} (\sphinxstyleliteralemphasis{\sphinxupquote{String}}) \textendash{} the input data for the checksum

\sphinxlineitem{Returns}
\sphinxAtStartPar
calculated checksum

\sphinxlineitem{Return type}
\sphinxAtStartPar
bitstring

\end{description}\end{quote}

\end{fulllineitems}


\sphinxAtStartPar
To create a ping icmp packet based on a packet id and size,
use the \sphinxcode{\sphinxupquote{create\_packet()}} function:
\index{create\_packet() (in module my\_ping)@\spxentry{create\_packet()}\spxextra{in module my\_ping}}

\begin{fulllineitems}
\phantomsection\label{\detokenize{index:my_ping.create_packet}}
\pysigstartsignatures
\pysiglinewithargsret
{\sphinxcode{\sphinxupquote{my\_ping.}}\sphinxbfcode{\sphinxupquote{create\_packet}}}
{\sphinxparam{\DUrole{n}{id}}\sphinxparamcomma \sphinxparam{\DUrole{n}{size}}}
{}
\pysigstopsignatures
\sphinxAtStartPar
Create a packet with a given id and of a given size
\begin{quote}\begin{description}
\sphinxlineitem{Parameters}\begin{itemize}
\item {} 
\sphinxAtStartPar
\sphinxstyleliteralstrong{\sphinxupquote{id}} (\sphinxstyleliteralemphasis{\sphinxupquote{string}}) \textendash{} id of the new packet

\item {} 
\sphinxAtStartPar
\sphinxstyleliteralstrong{\sphinxupquote{size}} (\sphinxstyleliteralemphasis{\sphinxupquote{int}}) \textendash{} size of the new packet

\end{itemize}

\sphinxlineitem{Returns}
\sphinxAtStartPar
the new icmp packet

\sphinxlineitem{Return type}
\sphinxAtStartPar
network packet

\end{description}\end{quote}

\end{fulllineitems}


\sphinxAtStartPar
To send a ping to a target ip address, use the \sphinxcode{\sphinxupquote{send\_ping()}} function:
\index{send\_ping() (in module my\_ping)@\spxentry{send\_ping()}\spxextra{in module my\_ping}}

\begin{fulllineitems}
\phantomsection\label{\detokenize{index:my_ping.send_ping}}
\pysigstartsignatures
\pysiglinewithargsret
{\sphinxcode{\sphinxupquote{my\_ping.}}\sphinxbfcode{\sphinxupquote{send\_ping}}}
{\sphinxparam{\DUrole{n}{target}}\sphinxparamcomma \sphinxparam{\DUrole{n}{packetsize}}}
{}
\pysigstopsignatures
\sphinxAtStartPar
Send a recieve a packet to a given target (of a given packetsize)
\begin{quote}\begin{description}
\sphinxlineitem{Parameters}\begin{itemize}
\item {} 
\sphinxAtStartPar
\sphinxstyleliteralstrong{\sphinxupquote{target}} (\sphinxstyleliteralemphasis{\sphinxupquote{string}}) \textendash{} the target destination

\item {} 
\sphinxAtStartPar
\sphinxstyleliteralstrong{\sphinxupquote{packetsize}} (\sphinxstyleliteralemphasis{\sphinxupquote{int}}) \textendash{} size of packets to be used as the pings

\end{itemize}

\sphinxlineitem{Returns}
\sphinxAtStartPar
status of this ping

\sphinxlineitem{Return type}
\sphinxAtStartPar
boolean

\end{description}\end{quote}

\end{fulllineitems}


\sphinxAtStartPar
To recieve a ping echo response, use the \sphinxcode{\sphinxupquote{receive\_ping()}} function:
\index{receive\_ping() (in module my\_ping)@\spxentry{receive\_ping()}\spxextra{in module my\_ping}}

\begin{fulllineitems}
\phantomsection\label{\detokenize{index:my_ping.receive_ping}}
\pysigstartsignatures
\pysiglinewithargsret
{\sphinxcode{\sphinxupquote{my\_ping.}}\sphinxbfcode{\sphinxupquote{receive\_ping}}}
{\sphinxparam{\DUrole{n}{sock}}\sphinxparamcomma \sphinxparam{\DUrole{n}{packet\_id}}\sphinxparamcomma \sphinxparam{\DUrole{n}{packetsize}}\sphinxparamcomma \sphinxparam{\DUrole{n}{timeout}\DUrole{o}{=}\DUrole{default_value}{10}}}
{}
\pysigstopsignatures
\sphinxAtStartPar
recieve a icmp ping echo response. the socket and packet\_id are provided, so we
know what to look for. A default timeout of 10 seconds is also applied, which should be plenty
for any address that is known to be online
\begin{quote}\begin{description}
\sphinxlineitem{Parameters}\begin{itemize}
\item {} 
\sphinxAtStartPar
\sphinxstyleliteralstrong{\sphinxupquote{sock}} (\sphinxstyleliteralemphasis{\sphinxupquote{socket}}) \textendash{} the socket that is prepared to accept the echo response

\item {} 
\sphinxAtStartPar
\sphinxstyleliteralstrong{\sphinxupquote{packet\_id}} (\sphinxstyleliteralemphasis{\sphinxupquote{string}}) \textendash{} id of the incoming echo response packet

\end{itemize}

\sphinxlineitem{Returns}
\sphinxAtStartPar
a tuple of the target address, rtt, and size of icmp packet

\sphinxlineitem{Return type}
\sphinxAtStartPar
tuple(string, double, int)

\end{description}\end{quote}

\end{fulllineitems}


\sphinxAtStartPar
To initialize the parser for a ping program,
use the \sphinxcode{\sphinxupquote{initialize\_parser()}} function:
\index{initialize\_parser() (in module my\_ping)@\spxentry{initialize\_parser()}\spxextra{in module my\_ping}}

\begin{fulllineitems}
\phantomsection\label{\detokenize{index:my_ping.initialize_parser}}
\pysigstartsignatures
\pysiglinewithargsret
{\sphinxcode{\sphinxupquote{my\_ping.}}\sphinxbfcode{\sphinxupquote{initialize\_parser}}}
{}
{}
\pysigstopsignatures
\sphinxAtStartPar
initialize the parser for this program. The only required argument is the ‘target’ which is the target ip address
to be pinged. Optional arguments include count, wait, packetsize, and timeout
\begin{quote}\begin{description}
\sphinxlineitem{Returns}
\sphinxAtStartPar
fully initialized parser

\sphinxlineitem{Return type}
\sphinxAtStartPar
parser

\end{description}\end{quote}

\end{fulllineitems}


\sphinxAtStartPar
To handle a timeout event, use the \sphinxcode{\sphinxupquote{timeout\_handler()}} function:
\index{timeout\_handler() (in module my\_ping)@\spxentry{timeout\_handler()}\spxextra{in module my\_ping}}

\begin{fulllineitems}
\phantomsection\label{\detokenize{index:my_ping.timeout_handler}}
\pysigstartsignatures
\pysiglinewithargsret
{\sphinxcode{\sphinxupquote{my\_ping.}}\sphinxbfcode{\sphinxupquote{timeout\_handler}}}
{\sphinxparam{\DUrole{n}{signum}}\sphinxparamcomma \sphinxparam{\DUrole{n}{frame}}}
{}
\pysigstopsignatures
\sphinxAtStartPar
handler for a program timeout. calls os.\_exit() to avoid raising an error, as this
can be called as part of an expected functionality

\end{fulllineitems}




\renewcommand{\indexname}{Index}
\printindex
\end{document}